\chapter{Conclusion}

\section{Deliverables} \label{sec: deliverables}

The deliverables will be the 3 components mentioned in the project systems in chapter \ref{ch: project-systems}.

\begin{itemize}
	\item Application Window
	\begin{itemize}
		\item \textbf{DS-1.1}: Plotting Library.
		\item \textbf{DS-1.2}: Signal Processing backend.
		\item \textbf{DS-1.3}: \ac{gui}.
		\item \textbf{DS-1.4}: \ac{gr} flow graph generator.
	\end{itemize}
	\item Signal Processing Library
	\begin{itemize}
		\item \textbf{DS-2.1}: \ac{ula}, \ac{ura}
		\item \textbf{DS-2.2}: \ac{mvdr}, \ac{lcmv}
		\item \textbf{DS-2.3}: Beamscan, \ac{mvdr}
	\end{itemize}
	\item \acf{oot} Modules
	\begin{itemize}
		\item \textbf{DS-3.1}: Array Response module
		\item \textbf{DS-3.2}: Steering Vector module
		\item \textbf{DS-3.3}: Beamforming module
		\item \textbf{DS-3.4}: \ac{doa} Estimation module
	\end{itemize}
\end{itemize}

\section{License}

All code created during this project will be released under the GNU General Public License v3 (GPLv3).

\section{Timeline}

This \acs{gsoc} project schedule spans over 22 weeks, from June-2 to November-10. The minimum guaranteed time spent per week roughly is,
$$ 3 \text{ hours} \times 6 \text{ days} + 1.5 \text{ hours on sunday} = 19.5 \frac{\text{hours}}{\text{week}}$$

The total time spent on the project,
$$ 3 \text{ hours} \times 6 \text{ days} \times 22 \text{ weeks} + 1.5 \text{ hours on sunday} \times 22 \text{ weeks} = 429 \text{ hours}$$

I am willing to put additional hours on holidays if necessary.

\begin{itemize}
	\item \textbf{May-8 to June-1}
	\begin{itemize}
		\item Completion of the milestone \hyperref[sec: milestones]{MS-1}.
	\end{itemize}
	\item \textbf{June-2 to June-16}
	\begin{itemize}
		\item Development of deliverables from \hyperref[sec: deliverables]{DS-2.1 to DS-2.3}.
		\item Interactions with mentors regarding the verification of the algorithms.
		\item Defining test cases and documentation.
		\item Verification.
	\end{itemize}
	\item \textbf{June-17 to July-1}
	\begin{itemize}
		\item Development of deliverables from \hyperref[sec: deliverables]{DS-3.1 and DS-3.2}.
		\item Interactions with mentors for the verification of the \ac{oot} modules in \ac{gr}.
		\item Defining test cases and documentation.
	\end{itemize}
	\item \textbf{July-2 to July-13}
	\begin{itemize}
		\item Verification.
		\item Presenting the developed algorithms to the mentors and getting feedback.
		\item Acting on the feedback.
		\item Do pending tasks if any.
		\item Completion of the milestone \hyperref[sec: milestones]{MS-3}.
	\end{itemize}
	\item \textbf{Midterm Evaluation Submission - July-14}
	\begin{itemize}
		\item Submission of deliverables from DS-2.1 to DS-3.2.
		\item Test case scripts and related documentation.
		\item Verification Report.
	\end{itemize}
	\item \textbf{July-13 to July-27}
	\begin{itemize}
		\item Development of deliverables from \hyperref[sec: deliverables]{DS-3.3 and DS-3.4}.
		 \item Interactions with mentors for the verification of the \ac{oot} modules in \ac{gr}.
		\item Defining test cases and documentation.
		\item Verification
	\end{itemize}
	\item \textbf{July-28 to August-3}
	\begin{itemize}
		\item Conduct research on the Qt Designer.
		\item Create a rough sketch of the graphical interface.
		\item Get feedback from mentors regarding the \ac{gui}.
	\end{itemize}
	\item \textbf{August-4 to August-25}
	\begin{itemize}
		\item Development of deliverables from \hyperref[sec: deliverables]{DS-1.1 to DS-1.3}
		\item Completion of milestone \hyperref[sec: milestones]{MS-2}.
		\item Meeting with the mentors regarding the developed \ac{ui} and its functionality.
	\end{itemize}
	\item \textbf{August-26 to September-1}
	\begin{itemize}
		\item Study about the ways to generate a \acl{gr} compatible flow graph from the user-defined parameters.
	\end{itemize}
	\item \textbf{September-2 to September-23}
	\begin{itemize}
		\item Meeting with mentors regarding the development of \hyperref[sec: deliverables]{DS-1.4} and its test cases.
		\item Development of the deliverable \hyperref[sec: deliverables]{DS-1.4}.
		\item Test cases and documentation.
		\item Verification.
		\item Completion of milestone \hyperref[sec: milestones]{MS-4}
	\end{itemize}
	\item \textbf{September-24 to September-30}
	\begin{itemize}
		\item Do any pending tasks or continue with schedule or 	break.
	\end{itemize}
	\item \textbf{October-1 to October-10}
	\begin{itemize}
		\item Integration of the plotting library and signal processing backend with the application window.
		\item Test cases and documentation.
		\item Verification.
	\end{itemize}
	\item \textbf{October-11 to October-31}
	\begin{itemize}
		\item Integration of the flow graph generator with the application window.
		\item Test cases and documentation.
		\item Fully integrated verification.
		\item Completion of the final milestone \hyperref[sec: milestones]{MS-5}.
	\end{itemize}
	\item \textbf{November-1 to November-9}
	\begin{itemize}
		\item Demonstration to the mentors.
		\item Submission of the project and documentations.
	\end{itemize} 
\end{itemize}


\section{About me}

\begin{itemize}
	\item Name: Rahul Rajeev Pillai
	\item Place of residence: Coimbatore, Tamil Nadu, India
	\item University: Amrita Vishwa Vidyapeetham
	\item Academic Background: Bachelors in Electrical and Computer Engineering (2024)
	\item Work Experience: 1 year
	\item Work Designation: Jr. Design Engineer
	\item Field: DSP/Communication
	\item g-mail: rahulpillairj@gmail.com
	\item github: \href{https://github.com/Ashborn-SM}{https://github.com/Ashborn-SM}
	\item Time zone: UTC+5:30
	
\end{itemize}

\section{A note to the mentors}
\begin{itemize}
	\item I am not a student but a full-time employee.
	\item For the interactions/communications with the mentors, it can be through g-mail, google-meet etc.
	\item This isn't my first open-source contributions. I started my journey way back in 2020 but had to stop for various reasons. Here's my previous contributions:  \href{https://github.com/MakeContributions/DSA/pulls?q=is%3Apr+is%3Aclosed+assignee%3AAshborn-SM+}{https://github.com/MakeContributions/DSA}
	\item For demonstrating my coding capabilities, here's the \href{https://github.com/Ashborn-SM/GSoC-Proposal/blob/main/plots.ipynb}{\textit{link}} to .ipynb file which contains the code for the various pattern plots seen in chapter \ref{ch: application-window}. The 3-D pattern seen in the title page is the array factor of the weights computed using \ac{lcmv}, which is also present in that file.
	\item The first two months of the timeline may look overcrowded but couple of them are already implemented in the .ipynb I shared in the above point. And also, I have experience developing couple of other Beamforming and \ac{doa} estimation algorithms in my work. So, in my opinion, it should not be taking any longer than what is projected for the completion of the project.
\end{itemize}

\section{Acknowledgement}

I've read and understood the GSoC Student Info and the manifest, and I agree to follow the rules, including  the three-strike policy. I’ll communicate regularly with my mentor, stay transparent about my progress, and follow community guidelines throughout the program.

\vspace{1cm}

{
	\centering
	\textit{Cyberspectrum is the best spectrum}\par
}